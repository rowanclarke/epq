\section{Uses of Machine Learning}

\subsection{Present Uses}

With growing data sets and dispensable hardware, machine learning is becoming increasingly more relevant in today's industry. Primary focuses are on detecting patterns in user's search history for advertisement, face recognition for social media sites, commuting predictions for traffic, surveillance footage for reporting crimes, and translation from and to different languages.
Any laborious and extensive task that seems simple can probably be and is automated using machine learning.

“It is difficult to think of a major industry that AI will not transform. This includes healthcare, education, transportation, retail, communications, and agriculture. There are surprisingly clear paths for AI to make a big difference in all of these industries.”--Andrew Ng

\subsection{Future Uses}

Machine learning has already in place for simple tasks for decades, however, only recently has hardware become so cheap and accessible that we can now use machine learning to accelerate complex algorithms such as fluid simulation, where the output doesn't have to be exactly accurate, as long as it is reasonable to the human eye. Another example would be in DLSS for ray trace rendering. It samples some of the rays to produce a low-quality noisy image, passes it through a network and produces a high-quality image.

Recently, there has been a scientific breakthrough in the field of biological engineering due to advancements in machine learning. Google's DeepMind software “AlphaFold 2”, has been able to accurately predict the folding of protein structures, a problem that has been around for 50 years. Researchers believe it will enable us to understand viruses, create drugs, and design vaccines. 