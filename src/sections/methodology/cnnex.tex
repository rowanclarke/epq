

If we want our output to be a horizontal edge detector for an image, we would use the matrix

\begin{equation}
\begin{bmatrix}
1 & 2 & 1 \\
0 & 0 & 0 \\
-1 & -2 & -1
\end{bmatrix}
\end{equation}

We can see that the nodes directly next to the target node are weighted more than those slightly farther away, and that the bottom row of the matrix is negative. 
This is so that if there is any discernible difference in colour horizontally it is amplified. To ensure that our output is positive, we must take the absolute of output. 

We apply the same logic vertically.
\begin{equation}
\begin{bmatrix}
1 & 0 & -1 \\
2 & 0 & -2 \\
1 & 0 & -1
\end{bmatrix}
\end{equation}

If we apply these filters as we did in \fig{fig:conv}, we achive the following images in \fig{fig:hedge} and \fig{fig:vedge}.

\begin{figure}[h]
\setlength{\unitlength}{0.14in}
\centering

\begin{tikzpicture}[scale=0.30](18,3) 

\node at (-0.8, 3) {$\begin{bmatrix}
1 & 2 & 1 \\
0 & 0 & 0 \\
-1 & -2 & -1
\end{bmatrix}$};
\node at (7.5, 0) [below] {input};
\node at (14, 0) [below] {$\bold{G}_y$};

\fill[black!] (4.5,3) rectangle (8.5, 5);
\fill[black!] (7.5,5) rectangle (9.5, 0);

\fill[black!] (12,2) rectangle (13, 5);
\fill[black!] (12,4) rectangle (15, 5);

\fill[black!75!white] (15,4) rectangle (16, 5);
\fill[black!75!white] (13,2) rectangle (14, 4);

\fill[black!25!white] (14,2) rectangle (15, 4);

\draw[step=1, very thin, xshift=0.5cm] (4, 0) grid (10, 6);
\draw[step=1, very thin] (12, 1) grid (16, 5);

\end{tikzpicture}
\caption{The horizontal edge detector applied to an image, $\bold{G}_y$}
\label{fig:hedge}
\end{figure}

\begin{figure}[h]
\setlength{\unitlength}{0.14in}
\centering

\begin{tikzpicture}[scale=0.30](18,3) 

\node at (0.2, 3) {$\begin{bmatrix}
1 & 0 & -1 \\
2 & 0 & -2 \\
1 & 0 & -1
\end{bmatrix}$};

\node at (7.5, 0) [below] {input};
\node at (14, 0) [below] {$\bold{G}_x$};

\fill[black!] (4.5,3) rectangle (8.5, 5);
\fill[black!] (7.5,5) rectangle (9.5, 0);

\fill[black!] (15, 1) rectangle (16, 4);
\fill[black!] (13, 1) rectangle (16, 2);

\fill[black!75!white] (15,4) rectangle (16, 5);
\fill[black!75!white] (13,2) rectangle (14, 4);

\fill[black!25!white] (14,2) rectangle (15, 4);

\draw[step=1, very thin, xshift=0.5cm] (4, 0) grid (10, 6);
\draw[step=1, very thin] (12, 1) grid (16, 5);

\end{tikzpicture}
\caption{The vertical edge detector applied to an image, $\bold{G}_x$}
\label{fig:vedge}
\end{figure}


%We can combine the output images into a final output image that represents all edges using the following equation

%\begin{equation}
%\bold{G} = \sqrt{{\bold{G}_x}^2+{\bold{G}_y}^2}
%\label{eq:edge}
%\end{equation}


%\begin{figure}[H]
%\setlength{\unitlength}{0.14in}
%\centering

%\begin{tikzpicture}[scale=0.30](18,3) 
%
%\node at (14, 0) [below] %{$\sqrt{{\bold{G}_x}^2+{\bold{G}_y}^2}$};
%
%\fill[black!71!white] (12,2) rectangle (13, 5);
%\fill[black!71!white] (12,4) rectangle (15, 5);
%
%\fill[black!71!white] (15, 1) rectangle (16, 4);
%\fill[black!71!white] (13, 1) rectangle (16, 2);
%
%\fill[black!75!white] (15,4) rectangle (16, 5);
%\fill[black!75!white] (13,2) rectangle (14, 4);
%
%\fill[black!25!white] (14,2) rectangle (15, 4);
%
%\draw[step=1, very thin] (12, 1) grid (16, 5);
%
%\end{tikzpicture}
%\caption{REDO The vertical combined with the %horizontal edge detector}
%\label{fig:gedge}
%\end{figure}