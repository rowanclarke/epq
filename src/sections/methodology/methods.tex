\section{Methodology}

\subsection{Aim}

Using our knowledge of \gls{model}ling the human brain, learning through the \gls{perceptron} \gls{model}, and back-propagation, we aim to evaluate and justify the extent to which any problem can be taught effectively to a machine such that it is able to give accurate results.

\subsection{Methods}

In order to do this, we must devise a method to measure each of the aspects of a neural network that directly affect the accuracy of the results. Once we have achieved this, we should determine and justify a method for avoiding these aspects for all problems, and then prove it through the use of outlining an optimisation. This should then be met with real-world observations to prove the usefulness of machine learning.

\subsection{Reliability}

The sources I cite are very well known and have been used in academic papers.

\cite{brain} Gordon M. Shepard is a pioneer in the area of neuroscience. He introduced the olfactory system, and used this as a method for \gls{model}ling \gls{dend}s in the brain amongst other areas. His education consists of a MD at Harvard in 1959, and a DPhil at Oxford in 1962. 

“By any measure this work is a classic...It will undoubtedly take its place as one of the most significant and comprehensive commentaries of our time on structure and function of nervous tissue.”--Electrocephalography and Clinical Neurophysiology

\cite{neuron} The experiment consisted of “five elderly men”, and their method produced accurate results.

\cite{hebb} Donald O. Hebb is a psychologist who specialises in understanding human learning. He had recieved 3 honorary doctorates in 1961, 1965, and 1975.

The Organisation of Behavior is seen as Donald Hebb's most affluential piece of work \gls{model}ling our view on neuropsychology.

\cite{learn} “The text is very strong with respect to presenting competing theories of learning phenomena. It is one of the best I’ve read in this regard.”--Thomas J. Faulkenberry

\cite{nnd} “This introductory neural network text targets senior undergraduates, first-year graduate students, and computing professionals who are exploring artificial neural networks.”--Stan J. Thomas
